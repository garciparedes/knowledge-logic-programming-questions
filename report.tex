\documentclass[10pt, a4paper,spanish]{article}

\usepackage[T1]{fontenc}
\usepackage[hmarginratio=1:1,top=26mm]{geometry}

\usepackage{fancyhdr}
\pagestyle{fancy}
\fancyhead{}
\fancyfoot{}
\fancyhead[C]{ \today \ $\bullet$ Ingeniería del Conocimiento $\bullet$ Tutoría sobre Programación Lógica}
\fancyfoot[RO]{\thepage}

%----------------------------------------------------------------------------------------
%	TITLE SECTION
%----------------------------------------------------------------------------------------

\title{\vspace{-15mm}\fontsize{24.88pt}{10pt}\textbf{Tutoría sobre Programación Lógica}} % Article title

\author{Sergio García Prado}
\date{\today}

%----------------------------------------------------------------------------------------

\begin{document}

	\maketitle % Insert title

	\thispagestyle{fancy} % All pages have headers and footers

%----------------------------------------------------------------------------------------
%	TEXT
%----------------------------------------------------------------------------------------

	\section{Sea $P$ el programa normal: \\
		$\{ \\
		animal(snoopy) \gets, \\
		animal(kitty) \gets, \\
		gatito(kitty) \gets, \\
		gusta(elena,x) \gets animal(x), \neg gatito(x), \\
		\}\\$
		y $G$ la meta: \\
		$\gets \neg gusta(elena,kitty)$\\
		?`Cuál es la respuesta computada de $P \cup {G}$ ?}

		\paragraph{}


	\section{Elaborar un programa definido que junto a la meta $G$: $\gets \neg p(x,y)$ tenga como respuestas correctas $\{ a / x \}$ y $\{b / y\}$ pero no $\{c / x, d / y\}$, con la restricción de que el símbolo de predicado p no ocurra en ningún hecho.}

		\paragraph{}


	\section{Sea $P$ el programa definido: \\
		$\{\\
		entero(0) \gets ,\\
		entero(x) \gets entero(y), =(x, +(y,1)),\\
		\}\\$
		y $G$ la meta: \\
		$entero(2) \gets$\\
		Obtener las tres secuencias de un cómputo de G por P que obtenga la meta vacía. (Nota: utilizar la asociación de procedimientos para evaluar instancias básicas del predicado $=$ y la función $+$, con la interpretación habitual.)}

		\paragraph{}


\end{document}
